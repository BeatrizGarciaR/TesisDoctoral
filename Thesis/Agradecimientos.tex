%Agradecimientos

\chapter{Acknowledgment}
\markboth{Acknowledgment}{}

%Agradezco principalmente a Dios que día a día me demuestra su infinito amor, que me ha dado el don de la vida y que está para mí en los momentos más difíciles y en los más felices. Agradezco que me permita ver lo hermoso de la vida y lo bello que es tener a personas que me amen a mi lado. 

%Agradezco a mis padres Jesús García Gámez y Beatriz Ramos Larralde que me han dado una gran vida. Gracias por darme lo más valioso que tienen: sus consejos, su sabiduría, su educación y su amor. Gracias por ser mi ejemplo y mi soporte, por mostrarme siempre que Dios está en nuestra familia, y por nunca darse por vencidos para que Kary y yo tengamos la mejor educación y seamos unas personas de bien; todo lo que han hecho por ambas lo recordaré por siempre y no terminaré de agradecérselos. Si no fuera por ustedes yo no habría llegado a ser quien soy, se merecen el reconocimiento de todos mis logros. Los amo a los dos. 

%Agradezco a mi hermana Karina Guadalupe García Ramos, quien fue mi apoyo incondicional cuando existieron días difíciles. A ti te debo mis momentos de tranquilidad al conversar cuando llegaba a casa agotada de un día de arduo trabajo de investigación. Gracias por ser también mi amiga. Te amo. 

%Agradezco a mi novio Francisco Gerardo Meza Fierro, quien durante estos dos años ha comprendido perfectamente lo que es estar en una maestría de este nivel y me ha dado todo su apoyo incondicional como compañero, amigo y novio. Gracias por ser mi pañuelo de lágrimas, mi opuesto perfecto y mi más grande motivador. Gracias por tu tiempo, tu cariño y por todas las ocasiones en que me apoyaste para seguir adelante cuando algo no iba bien con mi investigación. Gracias por haberme enseñado que cualquier cosa puede superarse si estamos juntos. Te amo. 

%Agradezco a toda mi familia por estar al pendiente de mi formación académica; a mis tíos, primos y mi abuela, que piensan en mi futuro y me dan consejos sobre mi futuro. Lamento no poder verlos tanto como antes, esto es un trabajo duro que necesita más tiempo del que creí, prometo comunicarme con ustedes más seguido, los quiero demasiado. 

%Agradezco a mis amigos y compañeros del posgrado que me dieron grandes consejos para seguir adelante y no dejarme vencer por las malas experiencias al inicio de mi investigación. Gracias por escucharme siempre y por esas reuniones en las que me han permitido conocerlos y ser parte de sus vidas, los considero como mi segunda familia. 

%Agradezco en especial a Víctor Domínguez que ha escuchado mis quejas, mis derrotas, mis frustraciones, mis alegrías y mis logros; gracias por ser un gran amigo y por alegrar mis días en Yalma. Agradezco también a Alberto y Delavy, porque nos han aceptado como parte de su vida a todos, son unas personas increíbles. Agradezco a Alan, Mayra, Gabriela, Yessica, David y Astrid, que más que compañeros de trabajo han sido para mí grandes personas que me han apoyado en algún momento de mi estancia en el posgrado tanto académicamente como personalmente. No tienen idea de cuánto los quiero y aprecio. Agradezco también a Citlali, Pablo, Eduardo y Daniel que gracias a su experiencia en el posgrado nos han apoyado a todos a seguir adelante. 

%Agradezco a mis amigos y compañeros de la Iglesia que siempre se preocupan por mí y me preguntan sobre mi maestría; a ellos que también han sido mi segunda familia desde hace años: Arturo, Airy, Obed, Paola, sí se pudo. 

Agradezco a Dios principalmente porque durante mi vida ha puesto en mi camino a las personas correctas y me ha regalado los momentos correctos para mejorar personal y académicamente. Agradezco que me permita tener salud y que me demuestre su amor directamente y a través de las personas que están a mi alrededor. 

Agradezco a mis padres Beatriz Ramos y Jesús García que en cada etapa de mi vida me han apoyado para seguir creciendo como mujer y como profesionista. Gracias por su acompañamiento en mi camino, por sus consejos, su amor y su protección, y por siempre preocuparse por el bienestar de mi hermana y mío para que tengamos una gran calidad de vida y valoremos cada instante. 

Agradezco a mi hermana Karina García que ha sido mi compañera de desvelos y distracciones y que me ha impulsado a recordar que la vida no es cien por ciento el trabajo si no que también hay que disfrutar de los buenos momentos en compañía de la familia y los amigos.

Agradezco a mis amigos desde maestría Alberto, Delavy, Gabriela, Mayra, Alan, Astrid, Citlali, Pablo y Yessica, quienes han sido parte importante de mi vida. Ellos han sido de gran apoyo cuando necesito hablar sobre mis problemas académicos porque son quienes más me entienden, y además de eso siempre están para mí en las buenas y en las malas. Agradezco también a mis amigos Obed, Arturo, Paola y Airy, que me han acompañado en muchos años de mi vida y quienes son para mí como una familia. Y especialmente agradezco a mi mejor amigo Victor Domínguez, con quien desahogo mis penas y con quien comparto mis alegrías día con día a pesar de vivir a ciento tres kilómetros de distancia, tú has sido mi soporte, mi paño de lágrimas y quien pone color a mi vida.

Agradezco a la Facultad de Ingeniería Mecánica y Eléctrica (FIME) y a la Universidad Autónoma de Nuevo León (UANL) por ofrecer un gran programa a\-ca\-dé\-mi\-co en el Posgrado en Ciencias en Ingeniería de Sistemas en el cual pude concluir una etapa académica que me ayudará a superarme a mí misma y por la beca que se me proporcionó para llevar a cabo mis estudios. Agradezco también las instalaciones que proporcionan a nuestro posgrado para que podamos tener un laboratorio en dónde trabajar y salones donde podamos tomar las materias que se nos imparten.

Agradezco a los Doctores que se encuentran en el Posgrado en Ciencias en Ingeniería en Sistemas por ser parte de mi crecimiento académico y por darme sus enseñanzas sobre lo que un investigador y un docente puede llegar a ser. En particular agradezco a los miembros de mi comité de tesis, por haber leído mi tesis y haberme apoyado en las correcciones de la misma. De manera especial agradezco a mi asesor Roger Z. Ríos Mercado y a mi coasesora Yasmín A. Ríos Solís que durante toda mi investigación estuvieron apoyándome y dirigiendo la misma. 

Agradezco al Consejo Nacional de Ciencia y Tecnología (CONACyT) por la beca de manutención que me fue otorgada con la cual pude facilitarme mis estudios al utilizar ese ingreso en transporte, comida e incluso en la investigación que durante cuatro años estuve realizando. 
