%%%%%%%%%%%%%%%%%%%%%
% Documento maestro %
%%%%%%%%%%%%%%%%%%%%%
\documentclass[table]{fime}
\usepackage[table,xcdraw]{xcolor}
\usepackage{natbib}

\usepackage{comment}
%\usepackage{soul,color}$•$
%%%%%%%%%%%%%%%%%%%%%%%%%%%%%%%%%%%%%%%%%%%
% Cargando paquetes y definiendo opciones %
%%%%%%%%%%%%%%%%%%%%%%%%%%%%%%%%%%%%%%%%%%%
% Aquí puedes cargar los paquetes que vas a usar. La clase
% fime ya incluye babel, inputenc, graphicx y los de la AMS.
% Cargar un paquete está a tu libertad (y responsabilidad).
\usepackage{hyperref}
    \hypersetup{breaklinks=true,colorlinks=true,
        linkcolor=black,citecolor=black,urlcolor=black}
\usepackage[Algorithm]{algorithm}
\usepackage[noend]{algpseudocode}
\usepackage{graphicx} % figuras
\usepackage{subfigure} % subfiguras
\usepackage{pdfpages}
\usepackage{multirow}
\usepackage{stackengine}
\usepackage{subfigure}
\usepackage{graphicx}
\usepackage{float}
\usepackage{subfig}
%\usepackage[table,xcdraw]{xcolor}
%\usepackage{verbatim}

%\usepackage[spanish,onelanguage]{algorithm2e} 
%%%%%%%%%%%%%%%%%%%%%
% Definiendo campos %
%%%%%%%%%%%%%%%%%%%%%
\def\titulo{Stochastic me\-tho\-do\-lo\-gies for locating and dispatching two types of ambulances with partial coverage}
\def\autor{Beatriz Alejandra García Ramos}
\def\matricula{1550385}
\def\grado{Doctorado en Ciencias en Ingeniería de Sistemas}
% En caso de que el grado tenga orientación o especialidad llenar el siguiente
% campo dejando un ESPACIO INICIAL, en caso contrario, dejar vacío
%\def\orientacion{ con orientación en Diseño y Análisis}
\def\orientacion{}
\def\fecha{2024} % Coloca el mes con mayúscula inicial

\def\asesor{Dr. Roger Z. Ríos Mercado}
\def\coasesor{Dra. Yasmín Á. Ríos Solis}
\def\revisorA{Dra. Iris Abril Martínez Salazar}
\def\revisorB{Dra. María Angélica Salazar Aguilar}
% En el caso de que tu tesis sea de doctorado activa la variable cambiándola a \doctoradotrue
% y define tus otros dos revisores
\newif\ifdoctorado\doctoradotrue
\def\revisorC{Dr. Vincent André Lionel Boyer}
\def\revisorD{Dra. Yajaira Cardona Valdés}
\def\revisorE{Dra. Irma Delia García Calvillo}
% El visto bueno siempre va
\def\vobo{}

%%%%%%%%%%%%%%%%%%%%%%%
% Inicia el documento %
%%%%%%%%%%%%%%%%%%%%%%%
\begin{document}

\frontmatter
\pagestyle{main}

%%% Incluye PortillasM si tu tesis es de Maestría
%%% y PortillasD si es de doctorado.
% Portadas (Maestría)

\def\uanl{Universidad Autónoma de Nuevo León}
\def\fime{Facultad de Ingeniería Mecánica y Eléctrica}
\def\depg{Subdirección de Estudios de Posgrado}
\def\bety{Subdirector de Estudios de Posgrado}
\def\snnl{San Nicolás de los Garza, Nuevo León}

%%%%%%%%%%%%%%%%%%%%%%%%
% Primer portada: UANL %
%%%%%%%%%%%%%%%%%%%%%%%%
\thispagestyle{empty}

\begin{scshape}
\begin{center}
	{\Large \uanl} \\[5mm]
	{\large \fime} \\[5mm]
	{\large \depg}
	\vskip 15mm
	\includegraphics[height=55mm]{uanl}
	\vskip 12mm
	\begin{tabular}{p{11cm}}
		\centering
		{\large \titulo}
	\end{tabular}
	\vskip 7mm
	{por}\\[7mm]
	{\large \autor}\\[7mm]
	{como requisito parcial para obtener el grado de}\\[3mm]
	\MakeUppercase{\grado}\\
	\orientacion
	\vfill
	\fecha
\end{center}
\end{scshape}

%%%%%%%%%%%%%%%%%%%%%%%%%
% Segunda portada: FIME %
%%%%%%%%%%%%%%%%%%%%%%%%%
\newpage
\thispagestyle{empty}

\begin{scshape}
\begin{center}
	{\Large \uanl} \\[5mm]
	{\large \fime} \\[5mm]
	{\large \depg}
	\vskip 16mm
	\includegraphics[height=50mm]{fime}
	\vskip 16mm
	\begin{tabular}{p{11cm}}
		\centering
		{\large \titulo}
	\end{tabular}
	\vskip 7mm
	{por}\\[7mm]
	{\large \autor}\\[7mm]
	{como requisito parcial para obtener el grado de}\\[3mm]
	\MakeUppercase{\grado}\\
	\orientacion
	\vfill
	\fecha
\end{center}
\end{scshape}

%%%%%%%%%%%%%%%%%%%%%%%%%%%%%
% Carta del comité de tesis %
%%%%%%%%%%%%%%%%%%%%%%%%%%%%%
%\begin{comment}
\newpage
\thispagestyle{empty}
\enlargethispage{8mm}

{\renewcommand{\baselinestretch}{1.1}\selectfont
\begin{center}\vspace*{-25mm}\hspace*{-10mm}
\begin{minipage}{170.5mm}
\hspace{-1.5mm}\includegraphics[height=20mm]{uanl}\hfill\raise1mm\hbox{\includegraphics[height=18.5mm]{fime}}
\hrule\vspace{0.5mm}
\scalebox{.5}{\MakeUppercase{\uanl}}\hfill\scalebox{.5}{\MakeUppercase{\fime}}\medskip
\end{minipage}
\vskip4mm{\sc\large\uanl\\\fime\\[3pt]\depg}\vskip6mm
\end{center}

%\begin{center}
%{\bf \large \uanl} \\
%{\bf \fime} \\
%{\bf \depg}
%\end{center}

%\vskip 4mm
\vskip 2mm

Los miembros del Comité de Tesis recomendamos que la Tesis \textit{\titulo}, realizada por el alumno \autor, con número de matrícula \matricula, sea aceptada para su defensa como requisito parcial para obtener el grado de \grado\orientacion .
\ifdoctorado{\vskip 10mm}\else{\vskip 8mm}\fi

\begin{center}
El Comité de Tesis\\
\ifdoctorado{\vskip 15mm}\else{\vskip 25mm}\fi

\ifdoctorado{%%%
\begin{tabular}{p{37mm}p{21mm}p{12mm}p{21mm}p{37mm}p{21mm}}
	\cline{1-2} \cline{4-5}
	\multicolumn{2}{c}{\asesor} & & \multicolumn{2}{c}{\coasesor} \\
	\multicolumn{2}{c}{Asesor}   & & \multicolumn{2}{c}{Co-Asesor}   \\[10mm]
	%\cline{2-4}
	%& \multicolumn{3}{c}{\asesor} & \\
	%& \multicolumn{3}{c}{Asesor}  & \\[15mm]
	\cline{1-2} \cline{4-5}
	\multicolumn{2}{c}{\revisorA} & & \multicolumn{2}{c}{\revisorB} \\
	\multicolumn{2}{c}{Revisor}   & & \multicolumn{2}{c}{Revisor}   \\[10mm]
	\cline{1-2} \cline{4-5}
	\multicolumn{2}{c}{\revisorC} & & \multicolumn{2}{c}{\revisorD} \\
	\multicolumn{2}{c}{Revisor}   & & \multicolumn{2}{c}{Revisor}   \\[10mm]
	\cline{2-4}
	& \multicolumn{3}{c}{\revisorE} & \\
	& \multicolumn{3}{c}{Revisor}  & \\[5mm]
	& \multicolumn{3}{c}{Vo. Bo.} & \\[7mm]
	\cline{2-4}
	& \multicolumn{3}{c}{\vobo}   & \\
	& \multicolumn{3}{c}{\bety}   & \\ &&&&
\end{tabular}
}\else{%%%
\begin{tabular}{p{37mm}p{21mm}p{12mm}p{21mm}p{37mm}}
	\cline{2-4}
	& \multicolumn{3}{c}{\asesor} & \\
	& \multicolumn{3}{c}{Director}  & \\[19mm]
	\cline{1-2} \cline{4-5}
	\multicolumn{2}{c}{\revisorA} & & \multicolumn{2}{c}{\revisorB} \\
	\multicolumn{2}{c}{\revisorC}   & & \multicolumn{2}{c}{\revisorD}   \\[2mm]
	& \multicolumn{3}{c}{Vo. Bo.} & \\[17mm]
	\cline{2-4}
	& \multicolumn{3}{c}{\vobo}   & \\
	& \multicolumn{3}{c}{\bety}   & \\ &&&&
\end{tabular}
}\fi%%%

\vfill

\snnl, \MakeLowercase{\fecha}

\end{center}
%\end{comment}

%\includepdf[noautoscale, fitpaper=true]{CCF_000001.pdf}
% Dedicatoria

\thispagestyle{empty}
\vspace*{17mm}

\begin{flushright}
\begin{itshape}

%A mis padres Beatriz y Jesús\\
%que me han apoyado siempre a cumplir mis metas.\bigskip\bigskip

%A mi hermana Karina\\
%que también ha sido mi amiga.\bigskip\bigskip

%A mi novio Gerardo\\
%que ha vivido conmigo los momentos más difíciles.\bigskip\bigskip

%A mi familia y amigos\\
%que son mi apoyo incondicional.

\end{itshape}
\end{flushright}


 
\tableofcontents
\listoffigures
\listoftables

%Agradecimientos

\chapter{Acknowledgment}
\markboth{Acknowledgment}{} 

Agradezco a Dios principalmente porque durante mi vida ha puesto en mi camino a las personas correctas y me ha regalado los momentos correctos para mejorar personal y académicamente. Agradezco que me permita tener salud y que me demuestre su amor directamente y a través de las personas que están a mi alrededor. 

Agradezco a mis padres Beatriz Ramos Larralde y Jesús García Gámez que en cada etapa de mi vida me han apoyado para seguir creciendo como mujer y como profesionista. Gracias por su acompañamiento en mi camino, por sus consejos, su amor y su protección, y por siempre preocuparse por el bienestar de mi hermana y mío para que tengamos una gran calidad de vida y valoremos cada instante. 

Agradezco a mi hermana Karina Guadalupe García Ramos que ha sido mi compañera de desvelos y distracciones y que me ha impulsado a recordar que la vida no es cien por ciento el trabajo si no que también hay que disfrutar de los buenos momentos en compañía de la familia y los amigos.

Agradezco a mis amigos desde maestría Alberto, Delavy, Gabriela, Mayra, Alan, Astrid, Citlali, Pablo y Yessica, quienes han sido parte importante de mi vida. Ellos han sido de gran apoyo cuando necesito hablar sobre mis problemas académicos porque son quienes más me entienden, y además de eso siempre están para mí en las buenas y en las malas. Agradezco también a mis amigos Obed, Arturo, y Paola, que me han acompañado en muchos años de mi vida y quienes son para mí como una familia. Y especialmente agradezco a mi mejor amigo Victor Domínguez, con quien desahogo mis penas y con quien comparto mis alegrías día con día a pesar de vivir a ciento tres kilómetros de distancia, tú has sido mi soporte, mi paño de lágrimas y quien pone color a mi vida.

Agradezco al Consejo Nacional de Humanidades Ciencias y Tecnología (CONAH\-CyT) por la beca de manutención que me fue otorgada con la cual pude facilitarme mis estudios al utilizar ese ingreso en transporte, comida e incluso en la investigación que durante cuatro años estuve realizando. 

Agradezco a los Doctores que se encuentran en el Posgrado en Ciencias en Ingeniería en Sistemas por ser parte de mi crecimiento académico y por darme sus enseñanzas sobre lo que un investigador y un docente puede llegar a ser. En particular agradezco a los miembros de mi comité de tesis, por haber leído mi tesis y haberme apoyado en las correcciones de la misma. De manera especial agradezco a mi asesor Roger Z. Ríos Mercado y a mi coasesora Yasmín A. Ríos Solís que durante toda mi investigación estuvieron apoyándome y dirigiendo la misma. 

Agradezco a la Facultad de Ingeniería Mecánica y Eléctrica (FIME) y a la Universidad Autónoma de Nuevo León (UANL) por ofrecer un gran programa a\-ca\-dé\-mi\-co en el Posgrado en Ciencias en Ingeniería de Sistemas en el cual pude concluir una etapa académica que me ayudará a superarme a mí misma y por la beca que se me proporcionó para llevar a cabo mis estudios. Agradezco también las instalaciones que proporcionan a nuestro posgrado para que podamos tener un laboratorio en dónde trabajar y salones donde podamos tomar las materias que se nos imparten.

%Resumen

\chapter{Abstract}
\markboth{Resumen}{}

{\setlength{\leftskip}{10mm}
\setlength{\parindent}{-10mm}

\autor.

Candidato para obtener el grado de \grado\orientacion.

\uanl.

\fime.

Título del estudio: \textsc{\titulo}.

\noindent Número de páginas: \pageref*{lastpage}.}

%%% Comienza a llenar aquí
\paragraph{Objetivos y método de estudio:}


\paragraph{Contribuciones y conclusiones:}


\bigskip\noindent\begin{tabular}{lc}
\vspace*{-2mm}\hspace*{-2mm}Firma del director: & \\
\cline{2-2} & \hspace*{1em}\asesor\hspace*{1em}
\end{tabular}

%\end{comment}




\mainmatter
\pagestyle{fime}

%%% Haz un documento para cada capítulo
\chapter{Introducción}



\section{Hipótesis}



\section{Objetivo}



\section{Estructura de la Tesis}



\chapter{Antecedentes y Revisión Bibliográfica}


\chapter{Problema de estudio}


\chapter{Resultados Experimentales}



\chapter{Conclusiones y Trabajo a Futuro}



\section{Conclusiones}



\section{Trabajo a Futuro}



\appendix
%%% Haz un documento para cada apéndice
\include{Apendice}

\backmatter
\pagestyle{main}

%%% Aquí va la bibliografía, puedes usar el entorno de LaTeX (thebibliography)
%%% o la herramienta BibTeX. En caso de que optes por BibTeX, puedes usar
%%% alguno de los archivos de estilo (mighelbib.bst o mighelnat.bst) incluidos
%%% en el paquete, cuyos diseños armonizan con el diseño de tesis provisto por
%%% fime.cls. Para muestra, basta un botón:

%\bibliographystyle{plain}
\bibliographystyle{plainnat}
\bibliography{Biblio}

\label{lastpage}
%Autobiografia

\chapter*{Resumen autobiográfico}
\thispagestyle{empty}

\begin{center}
\autor

Candidato para obtener el grado de\\
\grado\\
\orientacion\bigskip

\uanl\\
\fime\bigskip

Tesis:\\
\textsc{\large\titulo}
\end{center}\bigskip

%Aquí va tu historia

Nací el 20 de enero de 1995 en el municipio de San Nicolás de los Garza en el estado de Nuevo León. Mis padres Jesús García Gámez y Beatriz Ramos Larralde me han cuidado y educado desde mi nacimiento, al igual que a mi hermana Karina Guadalupe García Ramos. Concluí mis estudios como Licenciado en Matemáticas en junio del año 2017 en la Facultad de Ciencias Físico Matemáticas perteneciente a la Universidad Autónoma de Nuevo León. Obtuve mi grado como Maestro en Ingeniería de Sistemas en noviembre de 2019 en la Facultad de Ingeniería Mecánica y Eléctrica perteneciente a la Universidad Autónoma de Nuevo León, en donde también inicié mis estudios de Doctorado en Ingeniería en Sistemas en agosto del año 2020.

\end{document}
